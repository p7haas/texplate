\section{Definition Zensur} \label{definition_zensur}
Zunächst soll der Begriff \textit{Zensur} defininiert werden. \citeauthor{Otto1968} beschreibt 1968 Zensur \enquote{generell im Sinne der autoritären Kontrolle menschlicher Äußerungen}\autocite[3]{Otto1968}. Sie verknüpft dies mit dem lateinischen Begriff \textit{censura}, der \textit{Prüfung, Beurteilung} bedeutet \autocite[Vgl.][3]{Otto1968}.  \citeauthor{Biermann1988} wählt einen engeren Begriff der Zensur. Sie soll \enquote{die Gesamtheit institutionell vollzogener und strukturell manifestierter Versuche bezeichnen, durch legale -- oder durch unrechtmäßige -- Anwendung  von Zwang oder physischer Gewalt gegen Personen oder Sachen schriftliche Kommunikation zu kontrollieren, zu verhindern oder fremdzubestimmen}\autocite[3]{Biermann1988}. \citeauthor{Zelger1999} legt gegenüber \citeauthor{Biermann1988} den Schwerpunkt der Definition weniger auf den Modus als auf die ausführende Instanz; er definiert Zensur als \enquote{die Überwachung, Bewertung und gegebenenfalls das Verbot und die Kontrolle der Durchführung dieses Verbots [\ldots] von veröffentlichten Meinungsäußerungen jeglicher Art durch die in einem Bereich herrschenden [sic!] Klasse, Partei, Staatsführung, Kirche oder Interessensgruppe zur Verhinderung nichtkonformer oder unkontrollierter Meinungsbildung}\autocite[10\psq]{Zelger1999}.
\section{Motivation staatlicher Zensur}
Nachdem der Begriff der Zensur definiert ist, sollen die Motivationsgründe für staatliche Zensurmaßnahmen im Überblick dargestellt werden: Warum zensiert ein Staat den Zugriff auf bestimmte Informationen?
Im deutschen Grundgesetz wird mit Artikel 5 Absatz 1 festgelegt: \blockquote{\enquote{Jeder hat das Recht, seine Meinung in Wort, Schrift und Bild frei zu äußern und zu verbreiten und sich aus allgemein zugänglichen Quellen ungehindert zu unterrichten. Die Pressefreiheit und die Freiheit der Berichterstattung durch Rundfunk und Film werden gewährleistet. Eine Zensur findet nicht statt.}\footnote{S. Grundgesetz für die Bundesrepublik Deutschland (GG) vom 23. Mai 1949, BGBl. S. 1, zuletzt geändert durch Gesetz vom 28. 8. 2006, BGBl. I S. 2034}} In Absatz 2 des selben Artikels wird dieses Recht auf freien Zugang zu Informationen wieder eingeschränkt: \enquote{Diese Rechte finden ihre Schranken in den Vorschriften der allgemeinen Gesetze, den gesetzlichen Bestimmungen zum Schutze der Jugend und in dem Recht der persönlichen Ehre.}. \citeauthor{Rosenthal2003} nennen ergänzend hierzu extremistisches, \enquote{terroristisches und rechtsradikales Gedankengut, sowie Gewaltverherrlichung und Pornographie [\ldots] jugendgefährdender Schriften [\ldots] Billigung einer Straftat, Verunglimpfung der Staatssymbole}\autocite[16]{Rosenthal2003}. Daraus folgt, dass in Deutschland Inhalte verboten werden können, um geltendes Recht zu wahren und Personengruppen wie Kinder und Jugendliche davor zu schützen.
Da das Internet nicht nur in Deutschland, sondern international verwendet wird, haben auch andere Länder bzw. deren Regierungen ein Interesse daran, das Internet zu zensieren. Darunter fallen verschiedene Themen wie beispielsweise die politische Unterdrückung von Dissidenten, Menschenrechtsaktivisten (zum Beispiel in China, Iran). Auch religiöse Gründe können zu einer Zensur führen\autocite[Vgl][3f]{Warf2010}. 
\section{Technische Grundlagen}\label{technischeGrundlagen}
\subsection{Referenzmodelle} \label{referenzmodelle}
Im vorliegenden Text -- wie auch in der Literatur -- wird bei der Erläuterung technischer Methoden der staatlichen Internetzensur auf das \acr{OSI}-\autocite[Vgl.][1335\psqq]{Day1983} sowie das \acr{TCP/IP}-Referenzmodell Bezug genommen. Beide Modelle untergliedern die Kommunikation zwischen zwei Hosts (genauer: zwischen zwei Prozessen \autocite[Vgl.][1338]{Day1983})  in logische Schichten, die jeweils von Diensten der darunterliegenden Schicht Gebrauch machen, sie mit eigener spezifischer Funktionalität anreichern und der darüberliegenden Schicht mittels einer präzise definierten Schnittstelle zur Verfügung stellen \autocite[Vgl.][1338]{Day1983}. \citeauthor{Tanenbaum2010} vergleichen das Prinzip der Schichtung mit dem \textit{Information Hiding}-Konzept der objektorientierten Programmierung: Der zugrunde liegende Gedanke ist, dass eine bestimmte Software (oder Hardware) ihren Nutzern einen Dienst anbietet, die Details seines internen Zustands und seiner Algorithmen aber vor ihnen versteckt \autocite[Vgl.][29]{Tanenbaum2010}. Die Regeln und Konventionen, nach denen Schicht \textit{n} auf einem Host mit Schicht \textit{n} auf einem anderen Host kommuniziert, werden als Schicht-\textit{n}- oder Layer-\textit{n}-Protokoll bezeichnet \autocite[Vgl.][29]{Tanenbaum2010}. 
\subsection{OSI-Modell}
Das \acr{OSI}-Modell wurde ab 1978 als Reaktion auf den wachsenden Gebrauch von Rechnernetzwerken entwickelt und 1983 als internationaler Standard \acr{ISO} 7498 verabschiedet \autocite[Vgl.][1334]{Day1983}. Es definiert sieben Schichten (s. Abb \ref{grafik:osi-modell}), welche \citeauthor{Day1983} sinngemäß folgendermaßen beschreiben \autocite[Vgl.][1334]{Day1983}:
\begin{itemize}
\item Die Applikationsschicht (\textit{application layer}) als Schnittstelle zwischen Anwendungen und Netzwerk ist die einzige Schicht, die von Anwendungen direkt genutzt wird und stellt alle dazu notwendigen Mittel zur Verfügung. 
\item Die Präsentationsschicht (\textit{presentation layer}) dient der Abstrahierung von Unterschieden in der Darstellung von Daten, beispielsweise unterschiedlicher Zeichensätze.
\item Die Sitzungsschicht (\textit{session layer}) organisiert und strukturiert Interaktionen zwischen Applikationsprozessen. Sie stellt Methoden zur Kontrolle der Kommunikation bereit.
\item Die Transportschicht (\textit{transport layer}) befasst sich mit dem transparenten Austausch von Daten zwischen Endsystemen und stellt den darüberliegenden Schichten einheitliche Zugriffsmittel zur Verfügung.
\item Die Vermittlungsschicht (\textit{network layer}) bietet vom Übertragungsmedium unabhängigen Zugriff zum Netzwerk. Hier sind Routing und Weiterleitung zwischen miteinander verbundenen Netzwerken angesiedelt.
\item Die Sicherungsschicht (\textit{data-link layer}) beinhaltet Mittel, um Daten zwischen Netzwerkteilnehmern zu übertragen und dabei auftretetende physische Übertragungsfehler zu korrigieren.
\item Die Bitübertragungsschicht (\textit{physical layer}) spezifiziert mechanische und elektrische Standards zum Zugriff auf das physische Medium.
\end{itemize}
\begin{figure}[tp]
    \centering
    \includegraphics[width=0.9\textwidth]{osi-modell.pdf}\linebreak
    In Anlehnung an \textcite[42]{Tanenbaum2010}.
    \caption{Das OSI-Referenzmodell.}
    \label{grafik:osi-modell}
\end{figure}
\subsection{TCP/IP-Referenzmodell}\label{sec:TCP/IP}
Die im heutigen Internet weit verbreitete Protokollfamilie \acr{TCP/IP} wird in ein vierteiliges Schichtenmodell untergliedert (s. Abb. \ref{grafik:tcpip-modell}) \autocite[Vgl.][46]{Tanenbaum2010}. Gegenüber dem \acr{OSI}-Modell fasst \citeauthor{Holtkamp2002} die Funktionalitäten wie folgt zusammen \autocite[Vgl.][12\psq]{Holtkamp2002}:
\begin{itemize}
\item Applikationsschicht (\textit{application layer}): Entspricht den Schichten 5-7 des \acr{OSI}-Modells, umfasst beispielsweise die Protokolle \acr{HTTP} und \acr{SMTP}.
\item Transportschicht (\textit{transport layer}): Entspricht der gleichnamigen Schicht 4 des \acr{OSI}-Modells und enthält die Protokolle \acr{TCP} und \acr{UDP}.
\item Internetschicht (\textit{internet layer}): Entspricht Schicht 3 des \acr{OSI}-Modells und umfasst das \acr{IP}-Protokoll.
\item Netzwerkschicht (\textit{network layer}): Entspricht den Schichten 1 und 2 des \acr{OSI}-Modells. \acr{TCP/IP} verfügt über keine eigenen Protokolle auf dieser Schicht und nutzt statt dessen bestehende Technologien, wie beispielsweise Ethernet.
\end{itemize}

\begin{figure}[tp]
    \centering
    \includegraphics[width=0.8\textwidth]{tcpip-modell.pdf}\linebreak
    In Anlehnung an \textcite[46]{Tanenbaum2010}.
    \caption{Das TCP/IP-Referenzmodell.}
    \label{grafik:tcpip-modell}
\end{figure}
\subsection{Architektur des Internet} \label{sec:IP}
Das Internet kann auf Schicht 3 des \acr{OSI}-Modells als dezentrale Zusammenschaltung einzelner Netzwerke oder Autonomer Systeme verstanden werden \autocite[Vgl.][437]{Tanenbaum2010}. Daraus folgt, dass Start- und Endpunkt einer Kommunikation nicht unbedingt im selben Netzwerk liegen müssen, und dass es mehr als einen einzelnen Pfad vom Start zum Ziel geben kann. Das \acr{IP}-Protokoll wurde von Beginn an für diese Struktur entworfen \autocite[Vgl.][438]{Tanenbaum2010}. Seine Aufgabe ist es, einzelne Datenpakete vom Ursprungs- zum Zielort zu transportieren (nach dem \textit{best-effort-Prinzip}, d.h. ohne dabei die Ankunft zu garantieren), unabhängig von der Anzahl einzelner Netzwerke, die zwischen Ursprung und Ziel liegen \autocite[Vgl.][438]{Tanenbaum2010}. Um dies zu ermöglichen, werden die einzelnen Netzwerke über Router miteinander verbunden, deren Funktion es in diesem Kontext ist, anhand einer Adresstabelle ein ankommendes Paket entweder an einen weiteren Router bzw. zuletzt an das Zielsystem weiterzuleiten, oder aber das Paket zu verwerfen \autocite[Vgl.][342]{Tanenbaum2010}.

\subsection{Internet Protocol Version 4}
Das im Internet verbreitetste Vermittlungsprotokoll ist \acr{IPv4}. Es handelt sich dabei um ein paketvermitteltes Protokoll, dessen kleinste Einheit das so genannte \textit{Datagramm} ist. Ein \acr{IPv4}-Datagramm besteht aus \textit{Header} und \textit{Nutzlast}. Der Header besteht aus einem 20 Byte großen Teil konstanter Länge sowie einem optionalen Teil variabler Länge \autocite[439]{Tanenbaum2010}. Die in Abb. \ref{grafik:ipv4-header} hervorgehobenen Felder des Headers sollen hier kurz erklärt werden, da sie zum Verständnis späterer Ausführungen hilfreich sind.
Die Felder \textit{Identification}, \textit{MF} und \textit{Fragment offset} dienen der Erkennung und Rekonstruktion fragmentierter \acr{IP}-Pakete. Fragmentierung tritt beispielsweise auf, wenn ein 1600 Byte großes Paket auf seinem Weg durch das Internet eine Route passiert, die maximal 1500 Byte pro Paket übertragen kann. In diesem Fall wird das Paket in zwei Pakete passender Größe aufgeteilt, die Headerdaten entsprechend gesetzt und die beiden entstehenden Pakete auf ihren weiteren Weg geschickt. Der empfangende Host nutzt seinerseits die drei Felder im Header, um die Fragmente richtig zuzuordnen und in die richtige Reihenfolge zu bringen. Das Feld \textit{Time to live} enthält die maximale Anzahl von Routern, die ein Paket zwischen Start und Ziel passieren darf. Jeder Router dekrementiert das Feld, bevor er das Paket weiterleitet. Erreicht der Wert des Feldes Null, wird das Paket verworfen. Dies dient der Robustheit des gesamten Netzes. Gelangen Pakete aufgrund äußerer Umstände in eine Routingschleife, werden sie dennoch nicht beliebig oft weitergeleitet. Die beiden Felder \textit{Source address} und \textit{Destination address} enthalten die Quell- und Zieladresse des Pakets.
\begin{figure}[tp]
    \centering
    \includegraphics[width=1\textwidth]{ipv4-header.pdf}\linebreak
    In Anlehnung an \textcite[439]{Tanenbaum2010}.
    \caption{Der \acr{IPv4}-Header.}
    \label{grafik:ipv4-header}
\end{figure}
\subsection{Transmisson Control Protocol}
Das \acr{TCP} wurde gezielt entworfen, um eine zuverlässige Ende-zu-Ende-Übertragung von Byteströmen über eine prinzipbedingt unzuverlässige Zusammenschaltung einzelner Netzwerke zu ermöglichen \autocite[Vgl.][552]{Tanenbaum2010}. Die TCP-Protokollimplementierung des sendenden Hosts nimmt dazu Daten als Streams entgegen, teilt diese in maximal 64 KB große Stücke auf und sendet jedes dieser Stücke als einzelnes \acr{IP}-Datagramm \autocite[Vgl.][552]{Tanenbaum2010}. Sobald TCP-Datagramme auf dem Zielhost eingehen, werden die Stücke in der richtigen Reihenfolge zusammengesetzt und dem verarbeitenden Prozess zur Verfügung gestellt \autocite[Vgl.][552]{Tanenbaum2010}. Da das \acr{IP}-Protokoll im Gegensatz zu \acr{TCP} keine Garantien bezüglich der Zustellung einzelner Pakete gibt, ist es Aufgabe des \acr{TCP}-Protokolls, den Verlust einzelner Pakete zu erkennen und zu beheben bzw., falls dies nicht möglich ist, eine entsprechende Fehlermeldung bereitzustellen.
\acr{TCP} fügt \acr{IP}-Adressen einen weiteren Adressmechanismus hinzu -- die so genannten \textit{Ports}, welche durch eine 2 Byte breite Zahl mit einem Wertebereich von 0 bis 65535 dargestellt werden. Die Kombination aus \acr{IP}-Adresse und Port ergibt den Endpunkt einer TCP-Verbindung auf beiden Seiten. Ein TCP-\textit{Segment} besteht analog zum \acr{IP}-\textit{Datagramm} aus einem Header sowie der Nutzlast. TCP-Header bestehen aus einem 20 Byte großen Teil fester Größe sowie einem variablen Teil \autocite[Vgl.][556]{Tanenbaum2010}.
\subsection{User Datagram Protocol}
Den Abschluss der Protokollgrundlagen soll das \acr{UDP} bilden. Hierbei handelt es sich ebenfalls um ein Transportprotokoll, es ist allerdings wesentlich simpler aufgebaut als \acr{TCP}. \acr{UDP} ist im Gegensatz zu TCP verbindungslos; es macht keinerlei Zusicherung bezüglich der Zustellung und verzichtet auf eigene Mechanismen zur Erkennung und zum erneuten Versand verlorener Pakete. Wie TCP verwendet es 2 Byte große \textit{Ports} zur Adressierung. UDP-\textit{Segmente} bestehen aus einem 8 Byte großen Header fester Länge und der Nutzlast.