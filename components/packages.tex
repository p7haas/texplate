% Copyright (c) Patrick Haas. All rights reserved.
%
% Licensed under the BSD license. See LICENSE.txt
% in the project root for license information.

%% Sprachspezifika laden
% \usepackage[
%     english,
%     french,
%     ngerman,
%     headfoot=ngerman
% ]{babel}
\usepackage{polyglossia}
\setdefaultlanguage{german}
\setotherlanguages{english,french}
\gappto\captionsgerman{\def\abbreviationsname{Abkürzungen}}

%% hyperref aware footnote.sty
\usepackage{footnotehyper}

%% Bibliographie einrichten
%% Zu den Optionen: s. Biblatex-Dokumentation
\usepackage[
    bibstyle=authoryear,
    mergedate=false,            % Das Erscheinungsjahr immer zweimal ausgeben
    citestyle=authoryear-icomp,
    doi=false,
    eprint=true,
    url=false,
    isbn=false,
    date=year,
    urldate=long,
    %autocite=footnote,
    autocite=inline,
    hyperref=true,
    ibidpage=true,
    ibidtracker=constrict,
    pagetracker=true,
    language=auto,
    arxiv=abs,
    backend=biber,
    sortcites=true,
    sortlocale=auto,
    maxcitenames=2,      % im Dokumentenkörper ab dem 3. Verfasser abkürzen
    maxbibnames=1000,    % in der Bibliographie "nie"
]{biblatex}

\usepackage{url}
\setcounter{biburllcpenalty}{7000}
\setcounter{biburlucpenalty}{8000}

\usepackage{scrlayer-scrpage}
\pagestyle{scrheadings}
% \automark{section}

%% Angenommen, Sie wollen, dass Abschnittsüberschriften im Kolumnentitel ohne
%% Gliederungsnummer gesetzt werden, so ist das ganz einfach mit:
% \renewcommand*{\sectionmarkformat}{}

% \ihead[]{\headmark} 
\ihead[]{} 
\chead*{\pagemark}
\ohead*{}
\iftoggle{final_version}{%
    \cfoot*{}%
}{%
    \cfoot*{\textcolor{tol7q2}{DRAFT}}%
}

\iftoggle{TG_use_geometry}{
    \usepackage[
        left=\texplatemarginleft,
        right=\texplatemarginright,
        top=\texplatemargintop,
        bottom=\texplatemarginbottom,
        a4paper
    ]{geometry}
}

%% Caption-Format:
%% Erweiterte Optionen für Captions von Floats, entweder per caption:
\usepackage[
    position=below,
    % tableposition=below,
    % margin=15pt,
    width=.75\textwidth,
    format=hang,
    font={normalsize, sf, singlespacing},
    labelfont={normalsize, sf, it, singlespacing},
    labelsep=colon,
    skip=12pt plus 2pt minus 2pt
]{caption}
%% oder per KOMAscript:
% \setkomafont{caption}{\sffamily}
% \setkomafont{captionlabel}{\itshape}
% \usepackage{ragged2e}
% \setcaptionalignment[figure]{L} % L=linksbündig mit ragged2e
% \setcaptionalignment[table]{L}
% \setcapmargin{2em}

%% -----------------------------------------
%% Hilfe, Mathemann!
%% -----------------------------------------
\usepackage{mathtools}
\usepackage{lualatex-math}

%% Differentialkoeffizienten einfach schreiben
%\usepackage{diffcoeff}

%% Erweiterte amsmath-Environments
%\usepackage{empheq}

%% Environments für Optimierungsprobleme
%\usepackage{optidef}

%% LuaTeX einrichten
\usepackage[tuenc]{fontspec}
\usepackage[
    math-style=ISO,
    partial=upright,
    sans-style=upright
]{unicode-math}
\usepackage{luacode}

%% Schöne Grafiken
\usepackage{pgf,tikz}

%% Todo-Notizen
%\usepackage{todonotes}

%% Paket für Zeilennummern laden
%\usepackage{lineno}

%% Automatische Zwischenräume nach Abkürzungen (s. commands.tex)
\usepackage{xspace}

%% PDF-spezifische Optionen -> Grafiken und Hyperlinks im PDF
\usepackage{ifpdf}

%% Booktabs-Paket für schönere Tabellen
\usepackage{longtable}
\LTpre=0.1pt
\usepackage{booktabs}

% Correct order of tables after \paragraph or \subparagraph
\usepackage{etoolbox}
\makeatletter
\patchcmd\longtable{\par}{\if@noskipsec\mbox{}\fi\par}{}{}
\makeatother
% Allow footnotes in longtable head/foot
\IfFileExists{footnotehyper.sty}{\usepackage{footnotehyper}}{\usepackage{footnote}}
\makesavenoteenv{longtable}

%% tabularx-Paket für einstellbare Spaltenbreite
% \usepackage{tabularx}
% \usepackage{tabu}

% \usepackage{upquote}

%% Mehrspaltiger Satz
\usepackage{multicol}

%% pdfpages zum Einbinden externer PDFs mit \includepdf
%\usepackage{pdfpages}

%% enumitem zum Konfigurieren des vertikalen Spacing in Listen
\usepackage{enumitem}

%% SI-Einheiten
\usepackage{siunitx}

%% Floatende Grafiken etc erst nach ihrer ersten Referenzierung im Text anzeigen, siehe auch
%% http://stackoverflow.com/questions/547508/in-latex-is-there-a-way-to-put-a-float-automatically-after-where-it-is-first-re
% \usepackage{flafter}
