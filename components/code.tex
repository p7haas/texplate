% Copyright (c) Patrick Haas. All rights reserved.
%
% Licensed under the BSD license. See LICENSE.txt
% in the project root for license information.

%% ----------------------------------------------------------------------------
%% Einbinden des Pakets für Quelltextformatierung,
%% entweder Minted oder Listings
%% ----------------------------------------------------------------------------

%% Minted

%% Aufzählung der Listings pro section oder chapter
\usepackage[
    newfloat,
    section
]{minted}
% Für eine Liste verfügbarer Stylesheets:
% > pygmentize -L styles
\usemintedstyle{bw}

%% Bei newfloat=false
% \renewcommand{\listingscaption}{Quelltext}
% \renewcommand\listoflistingscaption{Quelltexte}

%% Bei newfloat=true
\SetupFloatingEnvironment{listing}{name=Quelltext}
\SetupFloatingEnvironment{listing}{listname=Quelltexte}

%% Globale Minted-Optionen
\setminted{
    autogobble=true,            %% Whitespace am Zeilenanfang entfernen
    mathescape,                 %% Mathe in Code erlauben
    linenos,                    %% Zeilennummerierung zeigen, linenos=false zum Ausschalten
    numbersep=5pt,              %% Abstand zwischen Zeilennummern und Zeile
    % fontsize=\footnotesize,     %% Schriftgröße
    % frame=leftline,             %% Rahmenstil
    frame=lines,                 %% Rahmenstil
    rulecolor=Gray,             %% Rahmenfarbe
    % xleftmargin=5pt,            %% Randbreite links
    % xrightmargin=5pt,           %% Randbreite rechts
    framesep=6pt                %% Abstand zwischen Rahmen und Inhalt
}

\renewcommand{\theFancyVerbLine}{%
    \sffamily \textcolor{Gray}{%
        \scriptsize%
        \addfontfeatures{Numbers={Uppercase,Monospaced}}{\arabic{FancyVerbLine}}}}

% Das Paket minted definiert die Umgebung minted so, dass sie am Ende einen vertikalen 
% Abstand einfügt. Der trifft dann mit dem vertikalen Abstand der Unterschrift (\caption)
% zusammen, wodurch dieser Abstand vergrößert wird. Damit ist auch der Weg klar, um das
% Problem zu lösen. Man muss irgendwie dafür sorgen, dass der Abstand, den minted am Ende
% einfügt, innerhalb der listing-Umgebung wieder entfernt wird. Das könnte man mit einem
% \unskip direkt nach \end{minted} lösen oder man ändert minted innerhalb listings generell
%
% WTF...
%
% Siehe: https://texwelt.de/wissen/fragen/8305/wie-kann-ich-den-abstand-der-caption-zu-einem-mit-minted-erstelltem-listing-verringern
\pretocmd{\listing}{%
    \apptocmd{\endminted}{\unskip}{}{\undefined}%
}{}{\undefined}

%% Minted-Shortcuts für im Dokument häufig
%% verwendete Programmiersprachen
%%
%% Genaue Beschreibungen der Parameter s. Minted-Doku
% \newminted{python}{
%    mathescape,			%% Mathe in Code erlauben
%    numbersep=5pt,		%% Abstand zwischen Zeilennummern und Zeile
%    gobble=8,			%% Menge der am Zeilenanfang entfernten Whitespaces
%    frame=leftline, 	%% Rahmenstil
%    linenos,			%% Zeilennummerierung, linenos=false zum Ausschalten
%    fontsize=\footnotesize, %% Schriftgröße
%    xleftmargin=5pt,	%% Randbreite links
%    xrightmargin=5pt,	%% Randbreite rechts
%    framesep=5pt    	%% Abstand zwischen Rahmen und Inhalt
% }

%% Wird dann so verwendet:
%
% \begin{csharpcode}
% 	int i = i*i
% \end{csharpcode}
%
%% Oder mit zusätzlichen Optionen:
%
% \begin{csharpcode} { gobble=4, frame=single }
% 	int i = i*i
% \end{csharpcode}
%


%% Listings

% \definecolor{bluekeywords}{rgb}{0.13,0.13,1}
% \definecolor{greencomments}{rgb}{0,0.5,0}
% \definecolor{redstrings}{rgb}{0.9,0,0}
 
% \usepackage{listings}
% \lstset{
%   language=Python,
%   showspaces=false,
%   showtabs=false,
%   breaklines=true,
%   showstringspaces=false,
%   breakatwhitespace=true,
%   escapeinside={(*@}{@*)},
%   numbers=left,
%   numberstyle=\tiny\color{black},
%   commentstyle=\color{greencomments},
%   keywordstyle=\color{bluekeywords}\bfseries,
%   stringstyle=\color{redstrings},
%   basicstyle=\small\ttfamily
% }
