% Copyright (c) Patrick Haas. All rights reserved.
%
% Licensed under the BSD license. See LICENSE.txt
% in the project root for license information.

% Microtype konfigurieren
\usepackage[
    tracking=true,
    final
]{microtype}


%% ----------------------------------------------------------------------------
%%
%% Zeilenabstand
%%
%% ----------------------------------------------------------------------------

%% Default ist 1,5facher Zeilenabstand
\usepackage[onehalfspacing]{setspace}

%% ----------------------------------------------------------------------------
%%
%% Floating-Environments
%%
%% ----------------------------------------------------------------------------

\deflength\textfloatsep{12.0pt plus 2.0pt minus 2.0pt}

%% ----------------------------------------------------------------------------
%%
%% Figure-Environments
%%
%% ----------------------------------------------------------------------------

\AtBeginEnvironment{figure}{\microtypesetup{activate=false} \singlespacing}

%% ----------------------------------------------------------------------------
%%
%% Listings-Environments
%%
%% ----------------------------------------------------------------------------

\AtBeginEnvironment{listings}{\microtypesetup{activate=false} \singlespacing}

%% ----------------------------------------------------------------------------
%%
%% Verbatim-Environments
%%
%% ----------------------------------------------------------------------------

\AtBeginEnvironment{verbatim}{\microtypesetup{activate=false}}

%% ----------------------------------------------------------------------------
%%
%% Tabellen
%%
%% ----------------------------------------------------------------------------

%% Tabular figures für Tabellen ;-)
\AtBeginEnvironment{longtable}{\addfontfeatures{Numbers={Lowercase,Monospaced}} \singlespacing}
\AtBeginEnvironment{tabular}{\addfontfeatures{Numbers={Lowercase,Monospaced}}}

%% ----------------------------------------------------------------------------
%%
%% Mikrotypographische Feinheiten
%%
%% ----------------------------------------------------------------------------

%% Th - Ligaturen ausschalten
% \DisableLigatures[T]{encoding = *, family = rm* }
\usepackage[german]{selnolig} % load selnolig w/o a language option
\nolig{Th}{T|h} % disable "Th" ligature globally

%% Aufrechte Klammern in kursivem Text
%\usepackage[biblatex=true]{embrac}
% Einstellungen für Skolar
%\AddEmph{[}{]}[.04em,0.8em]

%% ----------------------------------------------------------------------------
%%
%% Seite
%%
%% ----------------------------------------------------------------------------

%% Seitenzahl
\addtokomafont{pagenumber}{%
    \addfontfeatures{Numbers={Lining}}\sffamily\bfseries}

%% ----------------------------------------------------------------------------
%%
%% Überschriften
%%
%% ----------------------------------------------------------------------------

% %%  Abschnittsnummerierung in den linken Rand ausrücken.
% %%  Komascript definiert zur Ausgabe der Nummern pro
% %%  Abschnittsart NAME den Befehl \NAMEformat, der hier
% %%  umdefiniert wird. \theNAME gibt die Nummer selber aus,
% %%  \autodot fügt je nach Konfiguration einen Punkt an.
% %%  s. dazu auch scrguide Kap. 21
% \renewcommand*{\sectionformat}{%
%     \makebox[0pt][r]{\rmfamily\mdseries\upshape\thesection\autodot\enskip}}
% \renewcommand*{\subsectionformat}{%
%     \makebox[0pt][r]{\rmfamily\mdseries\upshape\thesubsection\autodot\enskip}}
% \renewcommand*{\subsubsectionformat}{%
%     \makebox[0pt][r]{\rmfamily\mdseries\itshape\thesubsubsection\autodot\enskip}}
% \renewcommand*{\paragraphformat}{%
%     \makebox[0pt][r]{\theparagraph\autodot\enskip}}
% \renewcommand*{\paragraphformat}{%
%     \makebox[0pt][r]{\thesubparagraph\autodot\enskip}}

% %% Auszeichnungen
% \addtokomafont{section}{%
%     %%\color{darkgray}\sffamily\bfseries\Large%
%     %\lsstyle\color{darkgray}\sffamily\bfseries\uppercase}
%     %\rmfamily\mdseries\upshape\lsstyle}
%     \addfontfeatures{Numbers={Uppercase}}\rmfamily\mdseries\upshape}
% \addtokomafont{subsection}{%
%     %\color{darkgray}\sffamily\bfseries\scshape}
%     \rmfamily\mdseries\scshape}
% \addtokomafont{subsubsection}{%
% %    \color{darkgray}\sffamily\bfseries%
%     \rmfamily\mdseries\itshape}
% %\addtokomafont{disposition}{%
% %    \color{darkgray}}

% %% Abstände um Überschriften ändern
% %% s. http://texwelt.de/wissen/fragen/10289/wie-andere-ich-die-abstande-uberunter-section-subsection-subsubsection
% %\RedeclareSectionCommand[%
%   %beforeskip=-1em,%
%   %afterskip=1sp]{section}
% %\RedeclareSectionCommand[%
%   %beforeskip=-1em,%
%   %afterskip=1sp]{subsection}
% %\RedeclareSectionCommand[%
%   %beforeskip=-1em,%
%   %afterskip=1sp]{subsubsection}
% \RedeclareSectionCommands[%
%   beforeskip=-1\baselineskip,%
%   afterskip=1sp%
% ]{section,subsection,subsubsection}

%% ----------------------------------------------------------------------------
%%
%% Fußnoten
%%
%% ----------------------------------------------------------------------------

%% Schriftart der Fußnotenmarke
\addtokomafont{footnotelabel}{%
    \sffamily\bfseries}

%% Ziffern linksbündig, Text darunter ausgerichtet
\deffootnote{1em}{1em}{%
  \makebox[1em][l]{\thefootnotemark}%
}

%% Ziffern in den Rand ausgerückt, Text darunter ausgerichtet
% \deffootnote{0em}{1em}{%
% %% \makebox[1.5em][l]{\addfontfeatures{Numbers={Proportional,Lining}}\textbf\thefootnotemark}%
%   \makebox[0pt][r]{\thefootnotemark\enskip}%
% }

%% Ziffern in den Rand ausgerückt, Text linksbündig
%% \deffootnote{0em}{1em}{%
%%   \makebox[0pt][r]{\textbf\thefootnotemark\enskip}%
%% }

%% Ziffern in den Rand ausgerückt, Text linksbündig
%% \deffootnote[1.5em]{0em}{1.5em}{%
%%   \makebox[1.5em][l]{\textbf\thefootnotemark}%
%% }

%% ----------------------------------------------------------------------------
%%
%% Definiert \textuppercase - Text in GROẞBUCHSTABEN mit Tracking
%%
%% ----------------------------------------------------------------------------
\makeatletter
\newcommand{\textuppercase}[1]{%
  {%
    \addfontfeatures{Numbers={Uppercase,Proportional}}%
    \microtypecontext{tracking=allcaps}%
    \lsstyle\MakeUppercase{#1}}}%

% \renewcommand{\sectionlinesformat}[4]{%
%     \Ifstr{#1}{section}{%
%         \@hangfrom{\hskip #2#3}{\textuppercase{#4}}%
%     }{%
%         \@hangfrom{\hskip #2#3}{#4}%
%     }%
% }
\makeatother

%% ----------------------------------------------------------------------------
%%
%% Literaturverzeichnis
%%
%% ----------------------------------------------------------------------------

%% Familienname in Kapitälchen
% \renewcommand{\mkbibnamefamily}[1]{\textsc{#1}}
