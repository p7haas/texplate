% Copyright (c) Patrick Haas. All rights reserved.
%
% Licensed under the BSD license. See LICENSE.txt
% in the project root for license information.

\PassOptionsToPackage{luatex}{graphicsx}
\PassOptionsToPackage{unicode=true,hidelinks}{hyperref} % options for packages loaded elsewhere
\PassOptionsToPackage{hyphens}{url}
% \PassOptionsToPackage{tuenc,Renderer=Harfbuzz}{fontspec}
\PassOptionsToPackage{tuenc}{fontspec}
\PassOptionsToPackage{cmyk,hyperref,usenames,dvipsnames}{xcolor}

%% Warnungen bei Kacklatex
%\usepackage[l2tabu, orthodox]{nag}

%% ----------------------------------------------------------------------------
%% KOMAScript
%%
%% fontsize = Schriftgröße,
%% BCOR = Bindekorrektur
%% titlepage = Titelei ausgeben
%% DIV = Textbreite
%% pagesize = Seitengröße
%% headings = Überschriftengröße
%% a4paper = Ausgabe in DIN A4
%% ----------------------------------------------------------------------------
\documentclass[
    fontsize=12,
    BCOR=15mm,
    titlepage=true,
    DIV=12,
    pagesize=auto,
    headings=small,
    headinclude=false,
    footinclude=true,
    footlines=2,
    captions=below,
    toc=listof,
    hidelinks,
    % tablecaptionsbelow,
    a4paper
]{scrartcl}

\KOMAoption{toc}{
  listof
}
\KOMAoptions{
  footnotes=multiple
}

\usepackage{scrhack}

%% ----------------------------------------------------------------------------
%% Generation of PDF/X and PDF/A compliant PDFs
%% See pdfx's documentation.
%% ----------------------------------------------------------------------------
%\usepackage[x-4]{pdfx} % for print
%\usepackage[x-4p]{pdfx} % for print, allow externally supplied color profile
\usepackage[a-1b]{pdfx} % for archival

%% ----------------------------------------------------------------------------
%% Basiskonfiguration einbinden
%% ----------------------------------------------------------------------------
\usepackage{etoolbox}

\newtoggle{final_version}
\newtoggle{TG_use_geometry}
\newtoggle{use_pandoc}

\IfFileExists{texplate-config.tex}%
    {\input{texplate-config.tex}}%
    {% Copyright (c) Patrick Haas. All rights reserved.
%
% Licensed under the BSD license. See LICENSE.txt
% in the project root for license information.

%% Vorabversion ein/aus
\settoggle{final_version}{false}

%% Geometry-Paket oder typearea für Satzspiegel verwenden
\settoggle{TG_use_geometry}{true}

%% Pandoc-Anpassungen ein/aus
\settoggle{use_pandoc}{false}

%% Titel der Arbeit
\newcommand{\texplatetitle}{Analyse und Visualisierung von Beziehungen zwischen Musikern auf der Grundlage von Musik-Metadaten unter Anwendung von Methoden der sozialen Netzwerkanalyse\xspace}
%% Untertitel der Arbeit
\newcommand{\texplatesubtitle}{\xspace}
%% Zweck der Arbeit
\newcommand{\texplatesubject}{Bachelor~Thesis zur Erlangung des Grads eines \textit{Bachelor~of~Science}\xspace}
%% Autor
\newcommand{\texplateauthor}{Patrick~Haas\xspace}
%% Betreuer
\newcommand{\texplatesupervisor}{Prof.~Dr.~Marcel~Endejan\xspace}
%% Ort
\newcommand{\texplatelocation}{Bonn\xspace}

%% Seitenränder für geometry-Paket
\newcommand{\texplatemarginleft}{4cm}%
\newcommand{\texplatemarginright}{2cm}%
\newcommand{\texplatemargintop}{3cm}%
\newcommand{\texplatemarginbottom}{2.5cm}%

}

\nottoggle{final_version}{%
    \KOMAoption{draft}{true}
}

% From https://github.com/jgm/pandoc/blob/master/data/templates/default.latex
\usepackage{graphicx}
% \makeatletter
% \def\maxwidth{\ifdim\Gin@nat@width>\linewidth\linewidth\else\Gin@nat@width\fi}
% \def\maxheight{\ifdim\Gin@nat@height>\textheight\textheight\else\Gin@nat@height\fi}
% \makeatother
% % Scale images if necessary, so that they will not overflow the page
% % margins by default, and it is still possible to overwrite the defaults
% % using explicit options in \includegraphics[width, height, ...]{}
% \setkeys{Gin}{width=\maxwidth,height=\maxheight,keepaspectratio}
% % Set default figure placement to htbp
% \makeatletter
% \def\fps@figure{htbp}
% \makeatother

%% ----------------------------------------------------------------------------
%% Kommandos zur Fehlersuche einbinden (nur in Vorabversionen)
%% ----------------------------------------------------------------------------
\nottoggle{final_version}{%
    % Copyright (c) Patrick Haas. All rights reserved.
%
% Licensed under the BSD license. See LICENSE.txt
% in the project root for license information.

%% ----------------------------------------------------------------------------
%%
%% Funktionen zur Fehlersuche
%%
%% ----------------------------------------------------------------------------


%% ----------------------------------------------------------------------------
%%
%% Environments in Rahmen setzen, um ihren Raumwunsch zu prüfen
%%
%% ----------------------------------------------------------------------------

% \BeforeBeginEnvironment{listing}{\begin{framed}}
% \AfterEndEnvironment{listing}{\end{framed}}

% \BeforeBeginEnvironment{minted}{\begin{framed}}
% \AfterEndEnvironment{minted}{\end{framed}}

% \BeforeBeginEnvironment{longtable}{\begin{framed}}
% \AfterEndEnvironment{longtable}{\end{framed}}

% \AtBeginEnvironment{figure}{\begin{framed}}
% \AtEndEnvironment{figure}{\end{framed}}
%
}

%% ----------------------------------------------------------------------------
%% Farbdefinitionen einbinden
%% ----------------------------------------------------------------------------
% Copyright (c) Patrick Haas. All rights reserved.
%
% Licensed under the BSD license. See LICENSE.txt
% in the project root for license information.

%% Erweiterte Farbdefinitionen
\usepackage[usenames, dvipsnames]{xcolor}

\definecolorset{RGB}{tol7q}{}{%
    1,51,34,136;%
    2,136,204,238%
}


%% ----------------------------------------------------------------------------
%% Pakete einbinden
%% ----------------------------------------------------------------------------
% Copyright (c) Patrick Haas. All rights reserved.
%
% Licensed under the BSD license. See LICENSE.txt
% in the project root for license information.

%% Sprachspezifika laden
% \usepackage[
%     english,
%     french,
%     ngerman,
%     headfoot=ngerman
% ]{babel}
\usepackage{polyglossia}
\setdefaultlanguage{german}
\setotherlanguages{english,french}
\gappto\captionsgerman{\def\abbreviationsname{Abkürzungen}}
% Namen für \autoref-Ziele
\gappto\captionsgerman{\def\figureautorefname{Abbildung}}
\gappto\captionsgerman{\def\tableautorefname{Tabelle}}
\gappto\captionsgerman{\def\partautorefname{Teil}}
\gappto\captionsgerman{\def\appendixautorefname{Anhang}}
\gappto\captionsgerman{\def\equationautorefname{Gleichung}}
\gappto\captionsgerman{\def\chapterautorefname{Kapitel}}
\gappto\captionsgerman{\def\sectionautorefname{Abschnitt}}
\gappto\captionsgerman{\def\subsectionautorefname{Abschnitt}}
\gappto\captionsgerman{\def\subsubsectionautorefname{Abschnitt}}
\gappto\captionsgerman{\def\Hfootnoteautorefname{Fußnote}}
\gappto\captionsgerman{\def\pageautorefname{Seite}}
% \providecaptionname{german}\figureautorefname{Abb.}

%% Bibliographie einrichten
%% Zu den Optionen: s. Biblatex-Dokumentation
\usepackage[
    bibstyle=authoryear,
    mergedate=false,            % Das Erscheinungsjahr immer zweimal ausgeben
    citestyle=authoryear-comp,
    doi=true,
    eprint=true,
    url=true,
    isbn=false,
    date=year,
    urldate=long,
    %autocite=footnote,
    autocite=inline,
    hyperref=true,
    % ibidpage=true,
    % useprefix=true,
    ibidtracker=constrict,
    pagetracker=true,
    language=auto,
    arxiv=abs,
    backend=biber,
    block=ragged,
    sorting=nyt,
    sortcites=true,
    sortlocale=auto,
    maxcitenames=2,      % im Dokumentenkörper ab dem 3. Verfasser abkürzen
    maxbibnames=1000,    % in der Bibliographie "nie"
]{biblatex}

\usepackage{xurl}
% \setcounter{biburllcpenalty}{7000}
% \setcounter{biburlucpenalty}{8000}
\urlstyle{rm}

\usepackage{scrlayer-scrpage}
\pagestyle{scrheadings}
% \automark{section}

%% Angenommen, Sie wollen, dass Abschnittsüberschriften im Kolumnentitel ohne
%% Gliederungsnummer gesetzt werden, so ist das ganz einfach mit:
% \renewcommand*{\sectionmarkformat}{}

\ihead[]{\headmark} 
% \ihead[]{} 
\chead*{}
\ohead*{\pagemark}
\iftoggle{final_version}{%
    \cfoot*{}%
}{%
    \cfoot*{\textcolor{tol_dark_cyan}{DRAFT}}%
}

\iftoggle{TG_use_geometry}{
    \usepackage[
        left=\texplatemarginleft,
        right=\texplatemarginright,
        top=\texplatemargintop,
        bottom=\texplatemarginbottom,
        a4paper
    ]{geometry}
}{}

%% Erweiterte Optionen für Captions von Floats, entweder per caption:
\usepackage[
    position=below,
    % tableposition=below,
    % margin=15pt,
    width=.90\textwidth,
    format=plain,
    % indention=2em,
    font={normalsize, sf, singlespacing},
    labelfont={normalsize, sf, it, singlespacing},
    labelsep=colon,
    skip=12pt plus 2pt minus 2pt
]{caption}
% Make caption package work with longtable
\makeatletter
\def\fnum@table{\tablename~\thetable}
\makeatother

\usepackage{subcaption}

%% oder per KOMAscript:
% \setkomafont{caption}{\sffamily}
% \setkomafont{captionlabel}{\scshape}
% \usepackage{ragged2e}
% \setcaptionalignment[figure]{L} % L=linksbündig mit ragged2e
% \setcaptionalignment[table]{L}
% \setcapmargin{1em}

%% -----------------------------------------
%% Fortlaufende Nummerierung für
%% figure und table
%% -----------------------------------------
\usepackage{chngcntr}
\counterwithout{figure}{chapter}
\counterwithout{table}{chapter}

%% -----------------------------------------
%% Hilfe, Mathemann!
%% -----------------------------------------
\usepackage{mathtools}
\usepackage{lualatex-math}

%% Differentialkoeffizienten einfach schreiben
%\usepackage{diffcoeff}

%% Erweiterte amsmath-Environments
%\usepackage{empheq}

%% Environments für Optimierungsprobleme
%\usepackage{optidef}

%% Relative Import
\usepackage{import}

%% LuaTeX einrichten
\usepackage[tuenc]{fontspec}
\usepackage[
    math-style=ISO,
    partial=upright,
    sans-style=upright
]{unicode-math}
\usepackage{luacode}

%% Schöne Grafiken
\usepackage{pgf,tikz}

%% Todo-Notizen
%\usepackage{todonotes}

%% Paket für Zeilennummern laden
%\usepackage{lineno}

%% Automatische Zwischenräume nach Abkürzungen (s. commands.tex)
\usepackage{xspace}

%% PDF-spezifische Optionen -> Grafiken und Hyperlinks im PDF
\usepackage{ifpdf}

%% Booktabs-Paket für schönere Tabellen
\usepackage{longtable}
\LTpre=0.1pt
\usepackage{booktabs}

\usepackage{pdflscape}
\usepackage{afterpage}

% Correct order of tables after \paragraph or \subparagraph
\usepackage{etoolbox}
\makeatletter
\patchcmd\longtable{\par}{\if@noskipsec\mbox{}\fi\par}{}{}
\makeatother
% Allow footnotes in longtable head/foot
\IfFileExists{footnotehyper.sty}{\usepackage{footnotehyper}}{\usepackage{footnote}}
\makesavenoteenv{longtable}

%% Bunte, gerahmte Boxen für alles
\usepackage{tcolorbox}

%% tabularx-Paket für einstellbare Spaltenbreite
% \usepackage{tabularx}
% \usepackage{tabu}

% \usepackage{upquote}

%% Mehrspaltiger Satz
\usepackage{multicol}

%% pdfpages zum Einbinden externer PDFs mit \includepdf
\usepackage{pdfpages}

%% enumitem zum Konfigurieren des vertikalen Spacing in Listen
\usepackage{enumitem}

%% SI-Einheiten
\usepackage{siunitx}

% \usepackage{float}
% \floatstyle{komabelow} % KOMA-Script definiert zwei eigene Stile für das float-Paket
% \restylefloat{table}
% \restylefloat{figure}

% \usepackage{floatrow}
% \floatsetup{style=plain}


%% Floatende Grafiken etc erst nach ihrer ersten Referenzierung im Text anzeigen, siehe auch
%% http://stackoverflow.com/questions/547508/in-latex-is-there-a-way-to-put-a-float-automatically-after-where-it-is-first-re
% \usepackage{flafter}


%% ----------------------------------------------------------------------------
%% Typographische Besonderheiten konfigurieren
%% ----------------------------------------------------------------------------
% Copyright (c) Patrick Haas. All rights reserved.
%
% Licensed under the BSD license. See LICENSE.txt
% in the project root for license information.

% Microtype konfigurieren
\usepackage[
    tracking=true,
    final
]{microtype}


%% ----------------------------------------------------------------------------
%%
%% Zeilenabstand
%%
%% ----------------------------------------------------------------------------

%% Default ist 1,5facher Zeilenabstand
\usepackage[onehalfspacing]{setspace}

%% ----------------------------------------------------------------------------
%%
%% Floating-Environments
%%
%% ----------------------------------------------------------------------------

\deflength\textfloatsep{12.0pt plus 2.0pt minus 2.0pt}

%% ----------------------------------------------------------------------------
%%
%% Figure-Environments
%%
%% ----------------------------------------------------------------------------

\AtBeginEnvironment{figure}{\microtypesetup{activate=false} \singlespacing}

%% ----------------------------------------------------------------------------
%%
%% Listings-Environments
%%
%% ----------------------------------------------------------------------------

\AtBeginEnvironment{listings}{\microtypesetup{activate=false} \singlespacing}

%% ----------------------------------------------------------------------------
%%
%% Verbatim-Environments
%%
%% ----------------------------------------------------------------------------

\AtBeginEnvironment{verbatim}{\microtypesetup{activate=false}}

%% ----------------------------------------------------------------------------
%%
%% Tabellen
%%
%% ----------------------------------------------------------------------------

%% Tabular figures für Tabellen ;-)
\AtBeginEnvironment{longtable}{\addfontfeatures{Numbers={Lowercase,Monospaced}} \singlespacing}
\AtBeginEnvironment{tabular}{\addfontfeatures{Numbers={Lowercase,Monospaced}}}

%% ----------------------------------------------------------------------------
%%
%% Mikrotypographische Feinheiten
%%
%% ----------------------------------------------------------------------------

%% Th - Ligaturen ausschalten
% \DisableLigatures[T]{encoding = *, family = rm* }
\usepackage[german]{selnolig} % load selnolig w/o a language option
\nolig{Th}{T|h} % disable "Th" ligature globally

%% ----------------------------------------------------------------------------
%%
%% Seite
%%
%% ----------------------------------------------------------------------------

%% Seitenzahl
\addtokomafont{pagenumber}{%
    \addfontfeatures{Numbers={Lining}}\sffamily\bfseries}

%% ----------------------------------------------------------------------------
%%
%% Überschriften
%%
%% ----------------------------------------------------------------------------

% %% Auszeichnungen
% \addtokomafont{section}{%
%     %%\color{darkgray}\sffamily\bfseries\Large%
%     %\lsstyle\color{darkgray}\sffamily\bfseries\uppercase}
%     %\rmfamily\mdseries\upshape\lsstyle}
%     \addfontfeatures{Numbers={Uppercase}}\rmfamily\mdseries\upshape}
% \addtokomafont{subsection}{%
%     %\color{darkgray}\sffamily\bfseries\scshape}
%     \rmfamily\mdseries\scshape}
% \addtokomafont{subsubsection}{%
% %    \color{darkgray}\sffamily\bfseries%
%     \rmfamily\mdseries\itshape}
% %\addtokomafont{disposition}{%
% %    \color{darkgray}}

% %% Abstände um Überschriften ändern
% %% s. http://texwelt.de/wissen/fragen/10289/wie-andere-ich-die-abstande-uberunter-section-subsection-subsubsection
% %\RedeclareSectionCommand[%
%   %beforeskip=-1em,%
%   %afterskip=1sp]{section}
% %\RedeclareSectionCommand[%
%   %beforeskip=-1em,%
%   %afterskip=1sp]{subsection}
% %\RedeclareSectionCommand[%
%   %beforeskip=-1em,%
%   %afterskip=1sp]{subsubsection}
% \RedeclareSectionCommands[%
%   beforeskip=-1\baselineskip,%
%   afterskip=1sp%
% ]{section,subsection,subsubsection}

%% ----------------------------------------------------------------------------
%%
%% Fußnoten
%%
%% ----------------------------------------------------------------------------

%% Schriftart der Fußnotenmarke
\addtokomafont{footnotelabel}{%
    \sffamily\bfseries}

%% Ziffern fett, Normalschrift, linksbündig, Text darunter ausgerichtet
\deffootnote{1em}{1em}{%
  \makebox[1em][l]{\thefootnotemark}%
}

%% ----------------------------------------------------------------------------
%%
%% Definiert \textuppercase - Text in GROẞBUCHSTABEN mit Tracking
%%
%% ----------------------------------------------------------------------------
\makeatletter
\newcommand{\textuppercase}[1]{%
  {%
    \addfontfeatures{Numbers={Uppercase,Proportional}}%
    \microtypecontext{tracking=allcaps}%
    \lsstyle\MakeUppercase{#1}}}%
\makeatother

%% ----------------------------------------------------------------------------
%%
%% Literaturverzeichnis
%%
%% ----------------------------------------------------------------------------

%% Familienname in Kapitälchen
% \renewcommand{\mkbibnamefamily}[1]{\textsc{#1}}


%% ----------------------------------------------------------------------------
%% Schriftsetup
%% ----------------------------------------------------------------------------
% % Copyright (c) Patrick Haas. All rights reserved.
%
% Licensed under the BSD license. See LICENSE.txt
% in the project root for license information.

\SetTracking[
    spacing = {45*,,},
    outer spacing = {30*,,}
]{
    encoding= *,
    shape = sc
}{10}

\SetTracking[
    context = allcaps,
    spacing = {400*,,},
    outer spacing = {300*,,}
]{
    encoding = *
}{50}

% Serifenschrift...
\setmainfont{LibertinusSerif-Regular.otf}[
    Ligatures={Common,TeX},
    % Scale=1, % 1 makes it about Times New Roman size
    BoldFont=LibertinusSerif-Bold.otf,
    ItalicFont=LibertinusSerif-Italic.otf,
    BoldItalicFont=LibertinusSerif-BoldItalic.otf,
    Numbers={OldStyle},
    Contextuals={Alternate}
]

% Serifenlose...
\setsansfont{LibertinusSans-Regular.otf}[
    Ligatures={Common, TeX},
    % Scale=1,
    BoldFont=LibertinusSans-Bold.otf,
    ItalicFont=LibertinusSans-Italic.otf,
    Numbers={OldStyle},
    Contextuals={Alternate}
]

% Nichtproportionale Schrift...
\setmonofont{Iosevka}[Scale=MatchLowercase]

% Matheschrift
\setmathfont{LibertinusMath-Regular.otf}[
    Ligatures=TeX,
    Scale=MatchLowercase
]

\setmathfontface\mathoper{LibertinusMath-Regular.otf}[
    Ligatures=TeX,
    Scale=MatchLowercase
]
\setoperatorfont\mathoper

%% Fancy Schauschrift
\newfontface\fancyheadline{LibertinusSerifDisplay-Regular.otf}[
    Ligatures=TeX
]

% Copyright (c) Patrick Haas. All rights reserved.
%
% Licensed under the BSD license. See LICENSE.txt
% in the project root for license information.

\SetTracking[
    spacing = {45*,,},
    outer spacing = {30*,,}
]{
    encoding= *,
    shape = sc
}{10}

\SetTracking[
    context = allcaps,
    spacing = {400*,,},
    outer spacing = {300*,,}
]{
    encoding = *
}{50}

% Serifenschrift...
\setmainfont{Skolar PE TEST}[
    Ligatures={Common,TeX},
    Scale=0.94, % 0.94 makes it about Times New Roman size
    Numbers={OldStyle},
    % Renderer=Harfbuzz,
    Contextuals={Alternate}
]

% Serifenlose...
\setsansfont{Skolar Sans PE TEST}[
    Ligatures={Common, TeX},
    Scale=0.94,
    Numbers={OldStyle},
    Contextuals={Alternate}
]

% Nichtproportionale Schrift...
\setmonofont{Iosevka}[Scale=MatchLowercase]

% Matheschrift
\setmathfont{Cambria Math}[
    Ligatures=TeX,
    Scale=MatchLowercase
]
% \setmathfont{SkolarPETEST-Regular.otf}[
%     Ligatures=TeX,
%     Scale=0.94,
%     range=up->up
% ]
% \setmathfont{SkolarPETEST-Italic.otf}[
%     Ligatures=TeX,
%     Scale=0.94,
%     range=it->up
% ]
% \setmathfont{SkolarPETEST-Bold.otf}[
%     Ligatures=TeX,
%     Scale=0.94,
%     range=bfup->up
% ]
% \setmathfont{SkolarPETEST-BoldItalic.otf}[
%     Ligatures=TeX,
%     Scale=0.94,
%     range=bfit->up
% ]
% \setmathfont{SkolarSansPECnTEST-Rg.otf}[
%     Ligatures=TeX,
%     Scale=0.94,
%     range=sfup->up
% ]
% \setmathfont{SkolarSansPECnTEST-Bd.otf}[
%     Ligatures=TeX,
%     Scale=0.94,
%     range=bfsfup->up
% ]
% \setmathfont{SkolarSansPECnTEST-It.otf}[
%     Ligatures=TeX,
%     Scale=0.94,
%     range=sfit->it
% ]
% \setmathfont{Cambria Math}[
%     range={
% 			"2207,  % Nabla
% 			"2202,  % Partial differential
% 			"1D715, % Partial differential italic
% 			tt,cal,bfcal,scr,bfscr,frak,bffrak,bb,bbit,bfsfit
% 		}
% ]
% \setmathfontface\mathoper{SkolarPETEST-Regular.otf}[
%     Ligatures=TeX,
%     Scale=0.94
% ]
\setmathfontface\mathoper{Cambria Math}[
    Ligatures=TeX,
    Scale=MatchLowercase
]
\setoperatorfont\mathoper

%% Fancy Schauschrift
\newfontface\fancyheadline{Skolar Sans PE Test Eb}[
    Ligatures=TeX
]

% % Copyright (c) Patrick Haas. All rights reserved.
%
% Licensed under the BSD license. See LICENSE.txt
% in the project root for license information.

\SetTracking[
    spacing = {45*,,},
    outer spacing = {30*,,}
]{
    encoding= *,
    shape = sc
}{10}

\SetTracking[
    context = allcaps,
    spacing = {400*,,},
    outer spacing = {300*,,}
]{
    encoding = *
}{50}

% Serifenschrift...
\setmainfont{Alegreya}[
    Ligatures={Common,TeX},
    Scale=1, % 1 makes it about Times New Roman size
    Numbers={OldStyle},
    BoldFont=Alegreya Bold,
    ItalicFont=Alegreya Italic,
    BoldItalicFont=Alegreya Bold Italic,
    SmallCapsFont=Alegreya SC
]

% Serifenlose...
\setsansfont{Alegreya Sans}[
    Ligatures={Common, TeX},
    Scale=1,
    BoldFont=Alegreya Sans Bold,
    ItalicFont=Alegreya Sans Italic,
    BoldItalicFont=Alegreya Sans Bold Italic,
    SmallCapsFont=Alegreya Sans SC,
    Numbers={OldStyle},
    Contextuals={Alternate}
]

% Nichtproportionale Schrift...
\setmonofont{Iosevka}[Scale=MatchLowercase]

% Matheschrift
\setmathfont{Cambria Math}[
    Ligatures=TeX,
    Scale=MatchLowercase
]

\setmathfontface\mathoper{Cambria Math}[
    Ligatures=TeX,
    Scale=MatchLowercase
]

\setoperatorfont\mathoper

%% Fancy Schauschrift
\newfontface\fancyheadline{Alegreya Sans Black}[
    Ligatures=TeX
]

% % Copyright (c) Patrick Haas. All rights reserved.
%
% Licensed under the BSD license. See LICENSE.txt
% in the project root for license information.

\SetTracking[
    spacing = {45*,,},
    outer spacing = {30*,,}
]{
    encoding= *,
    shape = sc
}{10}

\SetTracking[
    context = allcaps,
    spacing = {400*,,},
    outer spacing = {300*,,}
]{
    encoding = *
}{50}

% Serifenschrift...
\setmainfont{Minion Pro}[
    Ligatures={Common,TeX},
    %Scale=0.94, % 0.94 makes it about Times New Roman size
    Numbers={Lowercase,Proportional},
    % Contextuals={Alternate},
    UprightFeatures={
        SizeFeatures={ 
            {Size={-8.5},Font=MinionPro-Capt},
            {Size={8.5-13},Font=MinionPro-Regular},
            {Size={13-20},Font=MinionPro-Subh},
            {Size={20-},Font=MinionPro-Disp}
        },
    },
    BoldFeatures={
        SizeFeatures={ 
            {Size={-8.5},Font=MinionPro-BoldCapt},
            {Size={8.5-13},Font=MinionPro-Bold},
            {Size={13-20},Font=MinionPro-BoldSubh},
            {Size={20-},Font=MinionPro-BoldDisp}
        },
    },
    ItalicFeatures={
        SizeFeatures={ 
            {Size={-8.5},Font=MinionPro-ItCapt},
            {Size={8.5-13},Font=MinionPro-It},
            {Size={13-20},Font=MinionPro-ItSubh},
            {Size={20-},Font=MinionPro-ItDisp}
            },
    },
    BoldItalicFeatures={
        SizeFeatures={ 
            {Size={-8.5},Font=MinionPro-BoldItCapt},
            {Size={8.5-13},Font=MinionPro-BoldIt},
            {Size={13-20},Font=MinionPro-BoldItSubh},
            {Size={20-},Font=MinionPro-BoldItDisp}
        },
    },
]

% Serifenlose...
\setsansfont{Myriad Pro}[
    Ligatures={Common, TeX},
    %Scale=0.94,
    Numbers={Lowercase,Proportional},
    % Contextuals={Alternate},
    UprightFeatures={
        SizeFeatures={ 
            {Size={-10.5},Font=MyriadPro-SemiExt},
            {Size={10.5-},Font=MyriadPro-Regular},
        },
    },
    BoldFeatures={
        SizeFeatures={ 
            {Size={-10.5},Font=MyriadPro-BoldSemiExt},
            {Size={10.5-},Font=MyriadPro-Bold},
        },
    },
    ItalicFeatures={
        SizeFeatures={ 
            {Size={-10.5},Font=MyriadPro-SemiExtIt},
            {Size={10.5-},Font=MyriadPro-It},
        },
    },
    BoldItalicFeatures={
        SizeFeatures={ 
            {Size={-10.5},Font=MyriadPro-BoldSemiExtIt},
            {Size={10.5-},Font=MyriadPro-BoldIt},
        },
    },
]

% Nichtproportionale Schrift...
\setmonofont{Iosevka}[Scale=MatchLowercase]

% Matheschrift
\setmathfont{libertinusmath}[
    Ligatures=TeX,
    Scale=MatchLowercase
]

\setmathfontface\mathoper{libertinusmath}[
    Ligatures=TeX,
    Scale=0.94
]

\setoperatorfont\mathoper

%% Fancy Schauschrift
\newfontface\fancyheadline{MyriadPro-SemiboldSemiCn}[
    Ligatures=TeX
]

%% % Copyright (c) Patrick Haas. All rights reserved.
%
% Licensed under the BSD license. See LICENSE.txt
% in the project root for license information.

\SetTracking[
    spacing = {45*,,},
    outer spacing = {30*,,}
]{
    encoding= *,
    shape = sc
}{10}

\SetTracking[
    context = allcaps,
    spacing = {400*,,},
    outer spacing = {300*,,}
]{
    encoding = *
}{50}

% Serifenschrift...
\setmainfont{GaramondPremrPro.otf}[
    Ligatures={Common,TeX},
    %Scale=0.94, % 0.94 makes it about Times New Roman size
    Numbers={OldStyle},
    Contextuals={Alternate},
    UprightFeatures={
        SizeFeatures={ 
            {Size={-9},Font=GaramondPremrPro-Capt.otf},
            {Size={9-15},Font=GaramondPremrPro.otf},
            {Size={15-23},Font=GaramondPremrPro-Subh.otf},
            {Size={23-},Font=GaramondPremrPro-Disp.otf}
        },
    },
    BoldFeatures={
        SizeFeatures={ 
            {Size={-9},Font=GaramondPremrPro-SmbdCapt.otf},
            {Size={9-15},Font=GaramondPremrPro-Smbd.otf},
            {Size={15-23},Font=GaramondPremrPro-SmbdSubh.otf},
            {Size={23-},Font=GaramondPremrPro-SmbdDisp.otf}
        },
    },
    ItalicFeatures={
        SizeFeatures={ 
            {Size={-9},Font=GaramondPremrPro-ItCapt.otf},
            {Size={9-15},Font=GaramondPremrPro-It.otf},
            {Size={15-23},Font=GaramondPremrPro-ItSubh.otf},
            {Size={23-},Font=GaramondPremrPro-ItDisp.otf}
        },
    },
    BoldItalicFeatures={
        SizeFeatures={ 
            {Size={-9},Font=GaramondPremrPro-SmbdItCapt.otf},
            {Size={9-15},Font=GaramondPremrPro-SmbdIt.otf},
            {Size={15-23},Font=GaramondPremrPro-SmbdItSubh.otf},
            {Size={23-},Font=GaramondPremrPro-SmbdItDisp.otf}
        },
    },
]

% Serifenlose...
\setsansfont{Myriad Pro}[
    Ligatures={Common, TeX},
    %Scale=0.94,
    Numbers={OldStyle},
    Contextuals={Alternate},
    UprightFeatures={
        SizeFeatures={ 
            {Size={-10.5},Font=MyriadPro-SemiExt},
            {Size={10.5-},Font=MyriadPro-Regular},
        },
    },
    BoldFeatures={
        SizeFeatures={ 
            {Size={-10.5},Font=MyriadPro-BoldSemiExt},
            {Size={10.5-},Font=MyriadPro-Bold},
        },
    },
    ItalicFeatures={
        SizeFeatures={ 
            {Size={-10.5},Font=MyriadPro-SemiExtIt},
            {Size={10.5-},Font=MyriadPro-It},
        },
    },
    BoldItalicFeatures={
        SizeFeatures={ 
            {Size={-10.5},Font=MyriadPro-BoldSemiExtIt},
            {Size={10.5-},Font=MyriadPro-BoldIt},
        },
    },
]

% Nichtproportionale Schrift...
\setmonofont{Iosevka}[Scale=MatchLowercase]

% Matheschrift
\setmathfont{Cambria Math}[
    Ligatures=TeX,
    Scale=MatchLowercase
]

\setmathfontface\mathoper{Minion Math}[
    Ligatures=TeX,
    Scale=0.94
]
\setoperatorfont\mathoper

%% Fancy Schauschrift
\newfontface\fancyheadline{Garamond Premr Pro Smbd Disp}[
    Ligatures=TeX
]

% % Copyright (c) Patrick Haas. All rights reserved.
%
% Licensed under the BSD license. See LICENSE.txt
% in the project root for license information.

\SetTracking[
    spacing = {45*,,},
    outer spacing = {30*,,}
]{
    encoding= *,
    shape = sc
}{10}

\SetTracking[
    context = allcaps,
    spacing = {400*,,},
    outer spacing = {300*,,}
]{
    encoding = *
}{50}

% Serifenschrift...
\setmainfont{MetaSerifPro-Book}[
    Ligatures={Common,TeX},
    Scale=0.94, % 0.94 makes it about Times New Roman size
    Numbers={OldStyle},
    Contextuals={Alternate},
    BoldFont=MetaSerifPro-Bold,
    ItalicFont=MetaSerifPro-BookIta,
    BoldItalicFont=MetaSerifPro-BoldIta,
]

% Serifenlose...
\setsansfont{MetaOT-Book}[
    Ligatures={Common, TeX},
    Scale=0.94,
    BoldFont=MetaOT-Bold,
    ItalicFont=MetaOT-BookIta,
    BoldItalicFont=MetaOT-BoldIta,
    Numbers={OldStyle},
    Contextuals={Alternate}
]

% Nichtproportionale Schrift...
\setmonofont{Iosevka}[Scale=MatchLowercase]

% Matheschrift
\setmathfont{Cambria Math}[
    Ligatures=TeX,
    Scale=MatchLowercase
]

\setmathfontface\mathoper{Cambria Math}[
    Ligatures=TeX,
    Scale=MatchLowercase
]

\setoperatorfont\mathoper

%% Fancy Schauschrift
\newfontface\fancyheadline{MetaHeadOT-Bold}[
    Ligatures=TeX
]

%% % Copyright (c) Patrick Haas. All rights reserved.
%
% Licensed under the BSD license. See LICENSE.txt
% in the project root for license information.

\SetTracking[
    spacing = {45*,,},
    outer spacing = {30*,,}
]{
    encoding= *,
    shape = sc
}{10}

\SetTracking[
    context = allcaps,
    spacing = {400*,,},
    outer spacing = {300*,,}
]{
    encoding = *
}{50}

% Serifenschrift...
\setmainfont{ScalaPro}[
    Ligatures={Common,TeX},
    Scale=0.94, % 0.94 makes it about Times New Roman size
    Numbers={OldStyle},
    BoldFont=ScalaPro-Bold,
    ItalicFont=ScalaPro-Ita,
    BoldItalicFont=ScalaPro-BoldIta,
]

% Serifenlose...
\setsansfont{ScalaSansPro-Regular}[
    Ligatures={Common, TeX},
    Scale=0.94,
    BoldFont=ScalaSansPro-Bold,
    ItalicFont=ScalaSansPro-Italic,
    BoldItalicFont=ScalaSansPro-BoldItalic,
    Numbers={OldStyle},
    Contextuals={Alternate}
]

% Nichtproportionale Schrift...
\setmonofont{Iosevka}[Scale=MatchLowercase]

% Matheschrift
\setmathfont{Cambria Math}[
    Ligatures=TeX,
    Scale=MatchLowercase
]

\setmathfontface\mathoper{Cambria Math}[
    Ligatures=TeX,
    Scale=MatchLowercase
]
\setoperatorfont\mathoper

%% Fancy Schauschrift
\newfontface\fancyheadline{ScalaSansPro-Black}[
    Ligatures=TeX
]

%% % Copyright (c) Patrick Haas. All rights reserved.
%
% Licensed under the BSD license. See LICENSE.txt
% in the project root for license information.

\SetTracking[
    spacing = {45*,,},
    outer spacing = {30*,,}
]{
    encoding= *,
    shape = sc
}{10}

\SetTracking[
    context = allcaps,
    spacing = {400*,,},
    outer spacing = {300*,,}
]{
    encoding = *
}{50}

% Serifenschrift...
\setmainfont{FranziskaPro-Book}[
    Ligatures={Common,TeX},
    Scale=0.92, % 0.92 makes it about Times New Roman size
    Numbers={OldStyle},
    Contextuals={Alternate},
    BoldFont=FranziskaPro-Bold,
    ItalicFont=FranziskaPro-BookItalic,
    BoldItalicFont=FranziskaPro-BoldItalic,
]

% Serifenlose...
\setsansfont{MetaOT-Book}[
    Ligatures={Common, TeX},
    Scale=0.92,
    BoldFont=MetaOT-Bold,
    ItalicFont=MetaOT-BookIta,
    BoldItalicFont=MetaOT-BoldIta,
    Numbers={OldStyle},
    Contextuals={Alternate}
]

% Nichtproportionale Schrift...
\setmonofont{Iosevka}[Scale=MatchLowercase]

% Matheschrift
\setmathfont{Cambria Math}[
    Ligatures=TeX,
    Scale=MatchLowercase
]

\setmathfontface\mathoper{Cambria Math}[
    Ligatures=TeX,
    Scale=MatchLowercase
]

\setoperatorfont\mathoper

%% Fancy Schauschrift
\newfontface\fancyheadline{MetaHeadOT-Bold}[
    Ligatures=TeX
]

%% % Copyright (c) Patrick Haas. All rights reserved.
%
% Licensed under the BSD license. See LICENSE.txt
% in the project root for license information.

\SetTracking[
    spacing = {45*,,},
    outer spacing = {30*,,}
]{
    encoding= *,
    shape = sc
}{10}

\SetTracking[
    context = allcaps,
    spacing = {400*,,},
    outer spacing = {300*,,}
]{
    encoding = *
}{50}

% Serifenschrift...
\setmainfont{Lyon Text Regular}[
    Ligatures={Common,TeX},
    Scale=0.94, % 0.94 makes it about Times New Roman size
    Numbers={OldStyle},
    BoldFont=Lyon Text Semibold,
    ItalicFont=Lyon Text Regular Italic,
    BoldItalicFont=Lyon Text Semibold Italic,
]

% Serifenlose...
\setsansfont{ScalaSansPro-Regular}[
    Ligatures={Common, TeX},
    Scale=0.94,
    BoldFont=ScalaSansPro-Bold,
    ItalicFont=ScalaSansPro-Italic,
    BoldItalicFont=ScalaSansPro-BoldItalic,
    Numbers={OldStyle},
    Contextuals={Alternate}
]

% Nichtproportionale Schrift...
\setmonofont{Iosevka}[Scale=MatchLowercase]

% Matheschrift
\setmathfont{TeX Gyre Termes Math}[
    Ligatures=TeX,
    Scale=MatchLowercase
]

\setmathfontface\mathoper{TeX Gyre Termes Math}[
    Ligatures=TeX,
    Scale=MatchLowercase
]

\setoperatorfont\mathoper

%% Fancy Schauschrift
\newfontface\fancyheadline{Lyon Display Medium}[
    Ligatures=TeX
]

%% % Copyright (c) Patrick Haas. All rights reserved.
%
% Licensed under the BSD license. See LICENSE.txt
% in the project root for license information.

\SetTracking[
    spacing = {45*,,},
    outer spacing = {30*,,}
]{
    encoding= *,
    shape = sc
}{10}

%% \SetTracking[
%%     context = allcaps,
%%     spacing = {400*,,},
%%     outer spacing = {300*,,}
%% ]{
%%     encoding = *
%% }{50}

% Serifenschrift...
\setmainfont{TisaOT}[
    Ligatures={Common,TeX},
    Scale=0.92, % 0.92 makes it about Times New Roman size
    Numbers={OldStyle},
    Contextuals={Alternate},
    BoldFont=TisaOT-Bold,
    ItalicFont=TisaOT-Ita,
    BoldItalicFont=TisaOT-BoldIta,
]
\setsansfont{TisaSansPro}[
    Ligatures={Common, TeX},
    Scale=0.92,
    BoldFont=TisaSansPro-Bold,
    ItalicFont=TisaSansPro-Italic,
    BoldItalicFont=TisaSansPro-BoldItalic,
    Numbers={OldStyle},
    Contextuals={Alternate}
]

% Nichtproportionale Schrift...
\setmonofont{Iosevka}[Scale=MatchLowercase]

% Matheschrift
\setmathfont{Cambria Math}[
    Ligatures=TeX,
    Scale=MatchLowercase
]

\setmathfontface\mathoper{Cambria Math}[
    Ligatures=TeX,
    Scale=MatchLowercase
]
\setoperatorfont\mathoper

%% Fancy Schauschrift
\newfontface\fancyheadline{TisaSansPro-Bold}[
    Ligatures=TeX
]

%% % Copyright (c) Patrick Haas. All rights reserved.
%
% Licensed under the BSD license. See LICENSE.txt
% in the project root for license information.

\SetTracking[
    spacing = {45*,,},
    outer spacing = {30*,,}
]{
    encoding= *,
    shape = sc
}{10}

\SetTracking[
    context = allcaps,
    spacing = {400*,,},
    outer spacing = {300*,,}
]{
    encoding = *
}{50}

% Serifenschrift...
\setmainfont{Skolar PE TEST Light}[
    Ligatures={Common,TeX},
    Scale=0.95, % 0.95 makes it about Times New Roman size
    Numbers={OldStyle},
    Contextuals={Alternate},
    BoldFont=Skolar PE TEST Semibold,
    ItalicFont=Skolar PE TEST Light Italic,
    BoldItalicFont=Skolar PE TEST Semibold Italic,
]

% Serifenlose...
\setsansfont{SkolarSansPECnTEST-Rg.otf}[
    Ligatures={Common, TeX},
    Scale=0.95,
    BoldFont=SkolarSansPECnTEST-Bd.otf,
    ItalicFont=SkolarSansPECnTEST-It.otf,
    Numbers={OldStyle},
    Contextuals={Alternate}
]

% Nichtproportionale Schrift...
\setmonofont{Iosevka}[Scale=MatchLowercase]

% Matheschrift
\setmathfont{Cambria Math}[
    Ligatures=TeX,
    Scale=MatchLowercase
]

\setmathfontface\mathoper{Cambria Math}[
    Ligatures=TeX,
    Scale=MatchLowercase
]

\setoperatorfont\mathoper

%% Fancy Schauschrift
\newfontface\fancyheadline{SkolarSansPETEST-Eb.otf}[
    Ligatures=TeX
]

%% % Copyright (c) Patrick Haas. All rights reserved.
%
% Licensed under the BSD license. See LICENSE.txt
% in the project root for license information.

\SetTracking[
    spacing = {45*,,},
    outer spacing = {30*,,}
]{
    encoding= *,
    shape = sc
}{10}

\SetTracking[
    context = allcaps,
    spacing = {400*,,},
    outer spacing = {300*,,}
]{
    encoding = *
}{50}

% Serifenschrift...
\setmainfont{Andron Mega Corpus}[
    Ligatures={Common,TeX},
    Scale=0.94, % 0.94 makes it about Times New Roman size
    Numbers={OldStyle},
    BoldFont=Andron Mega Corpus SemiBold,
    ItalicFont=Andron Mega Corpus Italic,
    BoldItalicFont=Andron Mega Corpus SemiBold Italic,
    SmallCapsFont=Andron Mega Corpus SC
]

% Serifenlose...
\setsansfont{ScalaSansPro-Regular}[
    Ligatures={Common, TeX},
    Scale=0.94,
    BoldFont=ScalaSansPro-Bold,
    ItalicFont=ScalaSansPro-Italic,
    BoldItalicFont=ScalaSansPro-BoldItalic,
    Numbers={OldStyle},
    Contextuals={Alternate}
]

% Nichtproportionale Schrift...
\setmonofont{Iosevka}[Scale=MatchLowercase]

% Matheschrift
\setmathfont{Cambria Math}[
    Ligatures=TeX,
    Scale=MatchLowercase
]

\setmathfontface\mathoper{Cambria Math}[
    Ligatures=TeX,
    Scale=MatchLowercase
]

\setoperatorfont\mathoper

%% Fancy Schauschrift
\newfontface\fancyheadline{Lyon Display Medium}[
    Ligatures=TeX
]

% % Copyright (c) Patrick Haas. All rights reserved.
%
% Licensed under the BSD license. See LICENSE.txt
% in the project root for license information.

\SetTracking[
    spacing = {45*,,},
    outer spacing = {30*,,}
]{
    encoding= *,
    shape = sc
}{10}

\SetTracking[
    context = allcaps,
    spacing = {400*,,},
    outer spacing = {300*,,}
]{
    encoding = *
}{50}

% Serifenschrift...
\setmainfont{TheAntiquaB W5 Plain}[
    Ligatures={Common,TeX},
    Scale=0.94, % 0.94 makes it about Times New Roman size
    Numbers={Lowercase,Proportional},
    BoldFont=TheAntiquaB W7 Bold,
    ItalicFont=TheAntiquaB W5 Plain Italic,
    BoldItalicFont=TheAntiquaB W7 Bold Italic
]
% \setmainfont{Skolar PE TEST Light}[
%     Ligatures=TeX,
%     Scale=0.94,
%     BoldFont=Skolar PE TEST Semibold,
%     ItalicFont=Skolar PE TEST Light Italic,
%     BoldItalicFont=Skolar PE TEST Semibold Italic,
%     Numbers={OldStyle}
% ]
% \setmainfont{Minion Pro}[
%    Ligatures=TeX,
%    Numbers={OldStyle, Proportional}
% ]
% \setmainfont{Minion Pro}[
%     Ligatures=TeX,
%     Numbers={Proportional}
% ]
% \setmainfont{Adobe Caslon Pro}[
%     Ligatures=TeX,
%     Numbers={OldStyle, Proportional}
% ]
% \setmainfont{Adobe Garamond Pro}[
%     Ligatures=TeX,
%     Numbers={OldStyle, Proportional}
% ]
% \setmainfont{TeX Gyre Termes}[
%     Ligatures=TeX,
%     Numbers={Proportional}
% ]
% \setmainfont{TeX Gyre Pagella}[
%    Ligatures=TeX,
%    Scale=0.85,
%    Numbers={OldStyle, Proportional}
% ]
% \setmainfont{TeX Gyre Heros}[
%     Ligatures=TeX,
%     Numbers={Proportional}
% ]
% \setmainfont{Brill}[
%     Ligatures=TeX,
%     Numbers={Proportional}
% ]
% \setmainfont{Linux Libertine O}[
%     Ligatures=TeX,
%     Numbers={Proportional}
% ]
% \setmainfont{Vollkorn}[
%     Ligatures=TeX,
%     Scale=0.9,
%     Numbers={Proportional}
% ]
% \setmainfont{TeX Gyre Pagella}[
%     Ligatures=TeX,
%     Scale=0.9,
%     Numbers={Proportional}
% ]
% \setmainfont{Charis SIL}[
%     Ligatures=TeX,
%     Scale=0.9,
%     Numbers={Proportional}
% ]

% Serifenlose...
%\setsansfont{Myriad Pro}[
%    Ligatures=TeX,
%    Scale=MatchLowercase,
%    Numbers={OldStyle, Proportional}
%]
% \setsansfont{Myriad Pro}[
%     Ligatures=TeX,
%     Scale=MatchLowercase,
%     Numbers={Proportional}
% ]
\setsansfont{TheSansOsF-Plain}[
    Ligatures={Common, TeX},
    Scale=0.94,
    BoldFont=TheSansOsF-Bold,
    ItalicFont=TheSansOsF-PlainItalic,
    BoldItalicFont=TheSansOsF-BoldItalic,
    Numbers={Lowercase,Proportional},
    Contextuals={Alternate}
]
% \setsansfont{Aller}[
%     Ligatures=TeX,
%     Scale=MatchLowercase,
%     Numbers={OldStyle}
% ]
% \setsansfont{Linux Biolinum O}[
%     Ligatures=TeX,
%     Scale=MatchLowercase,
%     Numbers={Proportional}
% ]

% Nichtproportionale Schrift...
% \setmonofont{Inconsolata}[Scale=MatchLowercase]
% \setmonofont{Consolas}[Scale=MatchLowercase]
% \setmonofont{M+ 1mn}[Scale=MatchLowercase]
% \setmonofont{Sudo}[Scale=MatchLowercase]
\setmonofont{TheSansMonoCd W5Regular}[
    Scale=0.94,
    BoldFont=TheSansMonoCd W7Bold,
    ItalicFont=TheSansMonoCd W5Regular Italic,
    BoldItalicFont=TheSansMonoCd W7Bold Italic,
]

% Matheschrift
%\setmathfont{Asana Math}[
    %Ligatures=TeX,
    %Scale=MatchLowercase
%]
% \setmathfont{STIX Two Math}
\setmathfont{Cambria Math}[
    Ligatures=TeX,
    Scale=MatchLowercase
]
%% \setmathfont{SkolarPETEST-Regular.otf}[
%%     Ligatures=TeX,
%%     Scale=0.94,
%%     range=up->up
%% ]
%% \setmathfont{SkolarPETEST-Italic.otf}[
%%     Ligatures=TeX,
%%     Scale=0.94,
%%     range=it->up
%% ]
%% \setmathfont{SkolarPETEST-Bold.otf}[
%%     Ligatures=TeX,
%%     Scale=0.94,
%%     range=bfup->up
%% ]
%% \setmathfont{SkolarPETEST-BoldItalic.otf}[
%%     Ligatures=TeX,
%%     Scale=0.94,
%%     range=bfit->up
%% ]
%% \setmathfont{SkolarSansPECnTEST-Rg.otf}[
%%     Ligatures=TeX,
%%     Scale=0.94,
%%     range=sfup->up
%% ]
%% \setmathfont{SkolarSansPECnTEST-Bd.otf}[
%%     Ligatures=TeX,
%%     Scale=0.94,
%%     range=bfsfup->up
%% ]
%% \setmathfont{SkolarSansPECnTEST-It.otf}[
%%     Ligatures=TeX,
%%     Scale=0.94,
%%     range=sfit->it
%% ]
%% \setmathfont{Cambria Math}[
%%     range={
%% 			"2207,  % Nabla
%% 			"2202,  % Partial differential
%% 			"1D715, % Partial differential italic
%% 			tt,cal,bfcal,scr,bfscr,frak,bffrak,bb,bbit,bfsfit
%% 		}
%% ]
\setmathfontface\mathoper{Cambria Math}[
    Ligatures=TeX,
    Scale=MatchLowercase
]
\setoperatorfont\mathoper
%\setmathfont{Libertinus Math}[
%    Ligatures=TeX,
%    Scale=MatchLowercase
%]
% \setmathfont[range={cal,bfcal},StylisticSet=1]{STIX Two Math}
% \setmathfont[range={\mathup,\mathit,\mathbb,\mathbbit}]{Skolar PE TEST}
%% Fancy Schauschrift
\newfontface\fancyheadline{TheSansOsF-Bold}[
    Ligatures=TeX
]

%% % Copyright (c) Patrick Haas. All rights reserved.
%
% Licensed under the BSD license. See LICENSE.txt
% in the project root for license information.

\SetTracking[
    spacing = {45*,,},
    outer spacing = {30*,,}
]{
    encoding= *,
    shape = sc
}{10}

\SetTracking[
    context = allcaps,
    spacing = {400*,,},
    outer spacing = {300*,,}
]{
    encoding = *
}{50}

% Serifenschrift...
\setmainfont{TeX Gyre Termes}[
    Ligatures={Common,TeX},
    % Scale=1, % 1 makes it about Times New Roman size
    Numbers={OldStyle}
]

% Serifenlose...
\setsansfont{TeX Gyre Heros}[
    Ligatures={Common, TeX},
    % Scale=0.94,
    Numbers={OldStyle}
]

% Nichtproportionale Schrift...
\setmonofont{TeX Gyre Cursor}[Scale=MatchLowercase]

% Matheschrift
\setmathfont{TeX Gyre Termes Math}[
    Ligatures=TeX
]

\setmathfontface\mathoper{TeX Gyre Termes Math}[
    Ligatures=TeX
]
\setoperatorfont\mathoper

% Fancy Schauschrift
\newfontface\fancyheadline{TeX Gyre Heros Bold}[
    Ligatures=TeX
]


%% ----------------------------------------------------------------------------
%% Verzeichnisse umbenennen
%% ----------------------------------------------------------------------------
\renewcaptionname{german}{\contentsname}{Inhalt}
\renewcaptionname{german}{\listfigurename}{Abbildungen}
\renewcaptionname{german}{\listtablename}{Tabellen}
\renewcaptionname{german}{\figurename}{Abbildung}
\renewcaptionname{german}{\tablename}{Tabelle}

%% ----------------------------------------------------------------------------
%% Weitere Konfigurationsdateien laden
%% ----------------------------------------------------------------------------
% Copyright (c) Patrick Haas. All rights reserved.
%
% Licensed under the BSD license. See LICENSE.txt
% in the project root for license information.

%% Verschiedene selbstdefinierte Befehle

%% von Pandoc benötigt
\newcommand{\tightlist}{%
  \setlength{\itemsep}{0pt}\setlength{\parskip}{0pt}}

%% ordentliche Abkürzungen, Namen etc.
%% \newcommand{\Csharp}{C\nolinebreak{\textbf{\#}}\:}
%% \newcommand{\idR}{i.\,d.\,R. }
%% \newcommand{\iF}{i.\,F. }
%% \newcommand{\zB}{z.\,B. }

\newrobustcmd{\Csharp}{C\nolinebreak{\textbf{\#}}\:}
\newrobustcmd{\AbbreVaaO}{a.\,a.\,O.\xspace}
\newrobustcmd{\AbbreVdh}{d.\,h.\xspace}
\newrobustcmd{\AbbreVhrsgv}{hrsg.\,v.\xspace}
\newrobustcmd{\AbbreVidR}{i.\,d.\,R.\xspace}
\newrobustcmd{\AbbreViF}{i.\,F.\xspace}
\newrobustcmd{\AbbreVoO}{o.\,O.\xspace}
\newrobustcmd{\AbbreVoJ}{o.\,J.\xspace}
\newrobustcmd{\AbbreVua}{u.\,a.\xspace}
\newrobustcmd{\AbbreVzB}{z.\,B.\xspace}

% \makeatletter
% %% Hyperlinks für Abkürzungen entfernen
% \AtBeginDocument{%
%   \renewcommand*{\AC@hyperlink}[2]{#2}%
% }
% \makeatother

% Copyright (c) Patrick Haas. All rights reserved.
%
% Licensed under the BSD license. See LICENSE.txt
% in the project root for license information.

%% ----------------------------------------------------------------------------
%% Minted für Quelltextformatierung,
%% ----------------------------------------------------------------------------

%% Für scrartcl: Aufzählung der Listings pro section 
\usepackage[
    newfloat,
    section
]{minted}
% Für eine Liste verfügbarer Stylesheets:
% > pygmentize -L styles
\usemintedstyle{bw}

%% Bei newfloat=false
% \renewcommand{\listingscaption}{Quelltext}
% \renewcommand\listoflistingscaption{Quelltexte}

%% Bei newfloat=true
\SetupFloatingEnvironment{listing}{
    name=Quelltext,
    listname=Quelltexte,
    within=none
}

%% Globale Minted-Optionen
\setminted{
    autogobble=true,            %% Whitespace am Zeilenanfang entfernen
    mathescape,                 %% Mathe in Code erlauben
    linenos,                    %% Zeilennummerierung zeigen, linenos=false zum Ausschalten
    numbersep=5pt,              %% Abstand zwischen Zeilennummern und Zeile
    frame=lines,                 %% Rahmenstil
    rulecolor=Gray,             %% Rahmenfarbe
    framesep=6pt                %% Abstand zwischen Rahmen und Inhalt
}

\renewcommand{\theFancyVerbLine}{%
    \sffamily \textcolor{Gray}{%
        \scriptsize%
        \addfontfeatures{Numbers={Uppercase,Monospaced}}{\arabic{FancyVerbLine}}}}

% Das Paket minted definiert die Umgebung minted so, dass sie am Ende einen vertikalen 
% Abstand einfügt. Der trifft dann mit dem vertikalen Abstand der Unterschrift (\caption)
% zusammen, wodurch dieser Abstand vergrößert wird. Damit ist auch der Weg klar, um das
% Problem zu lösen. Man muss irgendwie dafür sorgen, dass der Abstand, den minted am Ende
% einfügt, innerhalb der listing-Umgebung wieder entfernt wird. Das könnte man mit einem
% \unskip direkt nach \end{minted} lösen oder man ändert minted innerhalb listings generell
%
% WTF...
%
% Siehe: https://texwelt.de/wissen/fragen/8305/wie-kann-ich-den-abstand-der-caption-zu-einem-mit-minted-erstelltem-listing-verringern
\pretocmd{\listing}{%
    \apptocmd{\endminted}{\unskip}{}{\undefined}%
}{}{\undefined}

%% Minted-Shortcuts für im Dokument häufig
%% verwendete Programmiersprachen
%%
%% Genaue Beschreibungen der Parameter s. Minted-Doku
% \newminted{python}{
%    mathescape,			%% Mathe in Code erlauben
%    numbersep=5pt,		%% Abstand zwischen Zeilennummern und Zeile
%    gobble=8,			%% Menge der am Zeilenanfang entfernten Whitespaces
%    frame=leftline, 	%% Rahmenstil
%    linenos,			%% Zeilennummerierung, linenos=false zum Ausschalten
%    fontsize=\footnotesize, %% Schriftgröße
%    xleftmargin=5pt,	%% Randbreite links
%    xrightmargin=5pt,	%% Randbreite rechts
%    framesep=5pt    	%% Abstand zwischen Rahmen und Inhalt
% }

%% Wird dann so verwendet:
%
% \begin{csharpcode}
% 	int i = i*i
% \end{csharpcode}
%
%% Oder mit zusätzlichen Optionen:
%
% \begin{csharpcode} { gobble=4, frame=single }
% 	int i = i*i
% \end{csharpcode}


%% ----------------------------------------------------------------------------
%% Quotation einrichten
%% ----------------------------------------------------------------------------
\iftoggle{final_version}{%
    \usepackage[strict=true,autostyle,german=guillemets,french=guillemets]{csquotes} % "Releasemodus"
}{%
    \usepackage[strict=false,debug=true,german=guillemets,french=guillemets]{csquotes} % "Debugmodus"
}
% \SetBlockThreshold{3}         % Blockquote ab 3 Zeilen
% \SetBlockEnvironment{quote}   % Name des Environments für Blockquotes (bspw. "quote")
\SetCiteCommand{\autocite}      % Das Zitationskommando (default: \cite)

%% hyperref immer nah am Ende laden!
\ifpdf
    % \usepackage[luatex]{graphicx}
    \usepackage[unicode=true,hidelinks]{hyperref}
    %\usepackage{hyperref}
    \hypersetup{
        bookmarksnumbered = true,
        colorlinks  = false,
        linkcolor   = black,
        citecolor   = Blue,
        urlcolor    = Blue,
        pdftitle    = {\texplatetitle},
        pdfsubject  = {\texplatesubject},
        pdfauthor   = {\texplateauthor}
    }
\else 
%    \usepackage{graphicx}
   \usepackage[hidelinks]{hyperref}    
   \hypersetup{
     pdftitle    = {\texplatetitle},
     pdfsubject  = {\texplatesubject},
     pdfauthor   = {\texplateauthor}
   }
\fi

%% glossaries-extra für Glossar und Abkürzungsverzeichnis
\usepackage[
    xindy={language=german-duden,codepage=utf8},
    nomain,
    toc=true,
    abbreviations,
    nonumberlist, % um Rückreferenzen zu unterdrücken
    nopostdot
]{glossaries-extra}
\usepackage{glossary-longragged}
\usepackage{glossaries-extra-stylemods}

\setglossarystyle{longragged}
\setabbreviationstyle[acronym]{long-short-sc}
\loadglsentries{components/glossar}
\makeglossaries

%% Datum setzen
\date{}

%% Einbinden der Bibliographiedatei aus Mendeley, Zotero, etc...
\addbibresource{library.bib}
%% Bibliographie erzeugen
\bibliography{library}{}

% Satzspiegel neuberechnen
\nottoggle{TG_use_geometry}{\recalctypearea}

%% Aufzählungen mittels itemize ohne Raum zwischen den Elementen
\setlist{noitemsep}
%% Aufzählungen mittels itemize ohne Raum zwischen den Elementen
%% und ohne Raum um die Aufzählung
%% \setlist{nosep}

\input{pandoc.tex}
